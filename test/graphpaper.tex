%%% examples for the GraphPaper package
%% Copyright (c) 2014, Daniel Meister <daniel@meisch.ch>
% to compile make sure the graphpaper.sty is correctly installed
% in you TeX distribution or in the same folder as this input file


\documentclass[11pt,a4paper,fleqn]{scrartcl}

\usepackage[a4paper,left=15mm,right=15mm,top=5mm,bottom=10mm,includehead,includefoot]{geometry}

\usepackage[utf8]{inputenc}
\usepackage{hyperref}
\usepackage{blindtext}
\usepackage{amsmath}

% include package and give box size as parameter
% currently only 4mm,5mm,6mm are supported
\usepackage[4mm]{graphpaper}

\title{GraphPaper Examples}
\author{Daniel Meister $\langle$\href{mailto:daniel@meisch.ch}{daniel@meisch.ch}$\rangle$}

\begin{document}
\maketitle
\centering

% default use: set width and height in number of boxes
\begin{graphpaper}{40}{10}
  Pythagorean theorem: \boldmath $a^2 + b^2 = c^2$\unboldmath\\[1\baselineskip]
  binomial coefficient: $\left(\begin{array}{c} n \\ k\end{array}\right) = \displaystyle\frac{n!}{k!\cdot\left(n-k\right)!}$
\end{graphpaper}

% gpconvert can be used to give lenghts in other units
\begin{graphpaper}{\gpconvert{13cm}}{\gpconvert{5cm}}
  \textbf{1.} $2 + 3 \cdot 5 = $\\
  \textbf{2.} $15 + 2 \cdot 7 = $
  \begin{align*}
  \text{\bf 1.} & & 2 + 3 \cdot 5 & = &&&&&\\
  \text{\bf 2.} & & 15 + 2 \cdot 7 & = &&&&&
  \end{align*}
  Complete all the above lines with the correct solution. Also indicate in what order you've evaluated the expressions.
\end{graphpaper}

% \gpconvert can also be used to scale to textwidth
% the height can be chosen automatically (integer number of boxes)
\begin{graphpaper}{\gpconvert{\textwidth}}{auto}
  A list of items:
  \begin{itemize}
  \item Item 1
  \item Item 2
  \end{itemize}
  \blindtext
\end{graphpaper}

\end{document}
